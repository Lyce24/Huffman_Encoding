\documentclass[letterpaper, 10pt]{article}

\usepackage{color}
\usepackage[top=0.75in, left=1in, right=1in, bottom=1.25in]{geometry}

\renewcommand{\mathbf}[1]{\mathbold{#1}}
\usepackage{amsmath, amssymb}
\usepackage{float}
\usepackage{marginnote}
\usepackage{tikz}
\usepackage{amsthm}
\usepackage{enumerate}
\usepackage{mathtools}
\usepackage{multicol}
\usepackage{epsfig}
\usepackage{tikz}
\usepackage{pgfplots}
\pgfplotsset{soldot/.style={color=black,only marks,mark=*}}\pgfplotsset{holdot/.style={color=black,fill=white,only marks,mark=*}}
\usepackage{polynom}
\usepackage{enumerate}
\usepackage{hyperref}

\newcommand\normal{\trianglelefteq}
\newcommand\Lefta{{\quad \quad \quad \quad \framebox[1.3\width]{4}}}
\newcommand\Leftb{{\quad \quad \quad \quad \framebox[1.3\width]{16}}}
\newcommand\Leftc{{\quad \quad \quad \quad \framebox[1.3\width]{10}}}
\newcommand\Leftd{{\quad \quad \quad \quad \framebox[1.3\width]{16}}}
\newcommand\Lefte{{\quad \quad \quad \quad \framebox[1.3\width]{15}}}

\def\Orb{\operatorname{Orb}}
\def\Stab{\operatorname{Stab}}

\begin{document}

  \pagestyle{empty}
  \thispagestyle{empty}

  \begin{center}
   \Large{Midterm II} \\
   \large{Math 332 -  Instructor: Marcus Robinson  } \\
  \end{center}

  \begin{center}
   Name: \underline{\hspace{4cm}}
  \end{center}

  \noindent \emph{\textbf{Instructions}:
  \begin{itemize}
      \item \textbf{Please pick the 72 hour period of your choosing to work on the exam.} Please submit your \TeX-ed solutions as a single pdf via Gradescope by 11:59 pm on Monday 4/4. 
      \item \textbf{If you plan to use office hours or come to the review session in class on Wednesday, please do not look at the exam before.}
      \item The exam will be open book, open note, open to anything posted on the Moodle. You are permitted to use a calculator for the purposes of verifying arithmetic. Using other people, the internet at large, or other people on the internet at large will be considered a violation of the honor principle.
      \item You may cite theorems proved in class, on homework and in the textbook. You should source your claims as best as possible (ie on the Worksheet from 2/15 we proved that...). You may NOT cite a theorem that trivializes the problem. 
      \item You should try to write excellent, well-edited proofs. You MUST justify your reasoning and failure to do so may result in your solution being marked-down. You should err on the side of over justifying. Partial credit will be given, so you are strongly encouraged to write your ideas about a problem even if you do not think you can write a rigorous proof. 
      \item If you have any questions or concerns, please email me as quickly as possible. 
  \end{itemize}
 } 
  
  
  
 \vspace{2cm}
 
  \begin{center}
\begin{figure}[H]
\centering
      \includegraphics[scale=.25]{img20.png}
      \caption{Just like me, these floofs know you are going to crush this exam!}
     \end{figure}
\end{center}

  
  \newpage


\noindent \textbf{Please answer 6 of the following 8 questions. Do not submit more than 6 answers.} \\

  
\noindent \textbf{1.} The group $GL_{2}(\mathbb{Z}_{2})$ acts on the set of column vector $X = \left\{ \begin{bmatrix}a \\ b\end{bmatrix} : a,b \in \mathbb{Z}_{2} \right\}$ via matrix multiplication. Let $v = \begin{bmatrix}1 \\ 1\end{bmatrix}$. Compute $\Orb_{v}$ and $\Stab_{v}$. \\

\noindent \textbf{2.} The group $U(32)$ is Abelian, so by the Fundamental Theorem of Abelian Groups is isomorphic to a direct product of cyclic groups. Express $U(32)$ in this way and construct an explicit isomorphism. \\

\noindent \textbf{3.} How many elements are in the conjugacy class of $(123)(789) \in S_{10}$. You can do this combinatorially or using Burnside's lemma. \\

\noindent \textbf{4.} Assume G is a finite Abelian group, where every element has order a power of 2. Prove that the order of G is a power of 2, and that the number of elements of order 2 in G is $2^{k}-1$, for some $k\geq 1$. \\

\noindent \textbf{5.} Assume that $G$ is a group acting on a set $X$. Fix some $ x\in X$. Prove that 
$$
\bigcap_{y \in \Orb_{x}} \Stab_{y} \normal G. 
$$

\noindent \textbf{6.} Use the Sylow Theorems to show there are no simple groups of order 75. \\

\noindent \textbf{7.} How many distinct ways are there to color the faces of a tetrahedron with $m$ colors up to rotation \footnote{Two colorings are considered distinct if there is no rotation of the tetrahedron that result in the same coloring of the faces} \footnote{There are 12 rotational symmetries of the tetrahedron!}.  \\

\noindent \textbf{8.} For each statement below, briefly explain why the statement is true or false.
\begin{itemize}
    \item The group $\mathbb{Z}_{1234567}$ is simple.
    \item Let $G$ and $H$ be non-trivial groups. Then $G \times H$ is never simple. 
    \item There exists a group $G$ and non-trivial homomorphism $\varphi: A_{5} \rightarrow G$ with $\ker(\varphi) = \langle (1\ 2\ 3) \rangle$. 
    \item There are $5$ pairwise non-isomorphic abelian groups of order $100$. 
\end{itemize}

\end{document}